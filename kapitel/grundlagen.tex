\section{Theoretische Grundlagen}

% TODO: Grundlagen ausformulieren und mit Quellen belegen

Dieses Kapitel erläutert die theoretischen Grundlagen, die für das Verständnis der entwickelten Middleware-Architektur erforderlich sind.

\subsection{SAP Business Technology Platform}

Die SAP \gls{btp} ist die zentrale Entwicklungs- und Integrationsplattform von SAP für Cloud-Anwendungen. Sie bietet eine Vielzahl von Services für die Entwicklung, Integration und Erweiterung von SAP-Anwendungen.

% TODO: BTP Services beschreiben, Cloud Foundry Environment

\subsubsection{Cloud Foundry Environment}

% TODO: Cloud Foundry erklären

\subsubsection{SAP Cloud ALM}

SAP \gls{calm} ist eine cloudbasierte Lösung für das Application Lifecycle Management. Es unterstützt Unternehmen bei der Planung, Implementierung und dem Betrieb von SAP-Lösungen.

% TODO: CALM Features beschreiben

\subsection{SAP Cloud Application Programming Model}

Das \gls{cap} ist ein Framework für die Entwicklung von Enterprise-Anwendungen auf der SAP \gls{btp}. Es basiert auf bewährten Open-Source-Technologien und bietet eine einheitliche Programmierabstraktion für verschiedene Datenbanken und Protokolle.

\subsubsection{Core Data Services}

\gls{cds} ist die deklarative Modellierungssprache des \gls{cap}. Mit \gls{cds} werden Datenmodelle und Services definiert.

\begin{lstlisting}[caption={Beispiel einer CDS-Service-Definition}, label={lst:cds-example}]
service BudgetService @(path: '/api/budget') {
    entity PSPElements {
        key psp      : String(50);
            name     : String(200);
            team     : String(100);
            budgetPT : Decimal(15, 2);
    }

    function getProjects() returns array of String;
    action startSession(project : String) returns { ... };
}
\end{lstlisting}

\subsubsection{Service-Implementierung in Node.js}

Die Geschäftslogik wird in JavaScript (Node.js) implementiert. CAP bietet dabei einen ereignisgesteuerten Ansatz mit Handlern für \gls{crud}-Operationen.

% TODO: Event Handler erklären

\subsection{OData Protocol}

Das \gls{odata} ist ein offener Standard für \gls{rest}ful \gls{api}s. Es definiert bewährte Praktiken für den Aufbau und die Nutzung von Web-APIs und wird von SAP als Standard für die Kommunikation zwischen Frontend und Backend verwendet.

\subsubsection{OData v4}

% TODO: OData v4 Features (Query Options, Actions, Functions)

Die wichtigsten Query-Optionen in OData v4 sind:
\begin{itemize}
    \item \texttt{\$filter} -- Filtern von Ergebnissen
    \item \texttt{\$select} -- Auswahl bestimmter Eigenschaften
    \item \texttt{\$orderby} -- Sortierung
    \item \texttt{\$top} und \texttt{\$skip} -- Paginierung
\end{itemize}

\subsection{Architekturmuster}

\subsubsection{Adapter Pattern}

Das Adapter-Pattern ist ein strukturelles Entwurfsmuster, das die Zusammenarbeit von Klassen mit inkompatiblen Schnittstellen ermöglicht \citep{gamma1994}. Es fungiert als Wrapper, der eine Schnittstelle in eine andere übersetzt.

% TODO: Adapter Pattern Diagramm einfügen

Im Kontext dieser Arbeit ermöglicht das Adapter-Pattern den Austausch der Datenquelle (Excel vs. OData-Service), ohne dass die darüberliegende Service-Schicht angepasst werden muss.

\subsubsection{Repository Pattern}

Das Repository-Pattern abstrahiert den Datenzugriff und stellt eine einheitliche Schnittstelle für die Geschäftslogik bereit \citep{fowler2002}.

% TODO: Repository Pattern erklären

\subsubsection{Schichtenarchitektur}

% TODO: Layer Architecture erklären

\subsection{Microsoft Graph API}

Die Microsoft \gls{graph} ermöglicht den programmgesteuerten Zugriff auf Microsoft 365-Dienste, einschließlich SharePoint und OneDrive. Für den Zugriff wird OAuth 2.0 verwendet.

% TODO: Authentication Flow beschreiben
