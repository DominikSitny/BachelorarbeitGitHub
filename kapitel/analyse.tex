\section{Analyse}

% TODO: Analyse ausformulieren

Dieses Kapitel analysiert die bestehende Situation und definiert die Anforderungen an die zu entwickelnde Middleware.

\subsection{Ist-Analyse}

\subsubsection{Excel-basierter Workflow}

Der aktuelle Workflow basiert auf einer zentralen Excel-Datei (\texttt{Budgetcontrolling.xlsx}), die über SharePoint geteilt wird. Die Datei integriert Daten aus verschiedenen Quellen mittels Power Query.

\begin{table}[H]
\centering
\caption{Sheets in Budgetcontrolling.xlsx}
\label{tab:excel-sheets}
\begin{tabularx}{\textwidth}{lXl}
\toprule
\textbf{Sheet} & \textbf{Zweck} & \textbf{Quelle} \\
\midrule
qry\_CALM\_Data & User Stories mit \gls{psp}-Zuordnung & Power Query \\
CADO & Zeitbuchungen (Ist-Stunden) & SAP-Export \\
PSP-Elemente & Budget pro \gls{psp} & Manuell \\
1. PSP-Zuordnung & Manuelle Zuordnungen & Manuell \\
2. PSP-Zuordnung & Fallback: Team $\rightarrow$ Timebox $\rightarrow$ \gls{psp} & Manuell \\
Reporting & Zusammenfassung pro Team & Berechnet \\
\bottomrule
\end{tabularx}
\end{table}

\subsubsection{Budget-Metriken}

Die folgenden Metriken werden für das Budget-Controlling verwendet:

\begin{table}[H]
\centering
\caption{Budget-Metriken und ihre Berechnung}
\label{tab:metriken}
\begin{tabularx}{\textwidth}{lXl}
\toprule
\textbf{Metrik} & \textbf{Beschreibung} & \textbf{Berechnung} \\
\midrule
Budget & Geplantes Budget & Summe aus PSP-Elemente \\
Actual & Bereits geleistete Arbeit & CADO Stunden / 8 \\
\gls{etc} & Noch zu leistende Arbeit & qry\_CALM\_Data (mit \gls{psp}) \\
\gls{eac} & Geschätzte Gesamtkosten & Actual + \gls{etc} \\
\bottomrule
\end{tabularx}
\end{table}

\subsubsection{Problemanalyse}

Die Analyse des bestehenden Workflows zeigt folgende Probleme:

\begin{enumerate}
    \item \textbf{Keine Echtzeitdaten}: Power Query wird nur bei manuellem Refresh aktualisiert
    \item \textbf{Keine API}: Andere Systeme können nicht auf die Daten zugreifen
    \item \textbf{Manuelle Zuordnungen}: \gls{psp}-Zuordnungen müssen manuell in Excel gepflegt werden
    \item \textbf{Fehlende Validierung}: Keine automatische Prüfung auf Konsistenz
    \item \textbf{Keine Historisierung}: Änderungen werden nicht nachvollziehbar protokolliert
\end{enumerate}

\subsection{Soll-Konzept}

Die Middleware soll die beschriebenen Probleme lösen und gleichzeitig einen einfachen Übergang zu einer service-orientierten Architektur ermöglichen.

% TODO: Soll-Konzept Diagramm einfügen

\subsection{Anforderungsanalyse}

\subsubsection{Funktionale Anforderungen}

\begin{table}[H]
\centering
\caption{Funktionale Anforderungen}
\label{tab:fa}
\begin{tabularx}{\textwidth}{lX}
\toprule
\textbf{ID} & \textbf{Anforderung} \\
\midrule
FA-01 & Das System muss Budget-Metriken pro Team berechnen können \\
FA-02 & Das System muss Budget-Metriken pro \gls{psp} berechnen können \\
FA-03 & Das System muss \gls{crud}-Operationen für \gls{psp}-Zuordnungen unterstützen \\
FA-04 & Das System muss Session-basiertes Projekt-Management unterstützen \\
FA-05 & Das System muss Team-Mismatch-Warnungen generieren können \\
FA-06 & Das System muss Excel-Daten von SharePoint lesen können \\
FA-07 & Das System muss Änderungen zurück in Excel schreiben können \\
\bottomrule
\end{tabularx}
\end{table}

\subsubsection{Nicht-funktionale Anforderungen}

\begin{table}[H]
\centering
\caption{Nicht-funktionale Anforderungen}
\label{tab:nfa}
\begin{tabularx}{\textwidth}{lX}
\toprule
\textbf{ID} & \textbf{Anforderung} \\
\midrule
NFA-01 & Die Backend-Implementierung muss austauschbar sein \\
NFA-02 & Die \gls{api} muss \gls{odata} v4-kompatibel sein \\
NFA-03 & Das System muss auf der SAP \gls{btp} deploybar sein \\
NFA-04 & Die \gls{api} muss für ein Fiori-Frontend geeignet sein \\
NFA-05 & Das System muss ohne lokale Datenbankinstanz funktionieren \\
\bottomrule
\end{tabularx}
\end{table}

\subsection{Abgrenzung}

Folgende Aspekte sind nicht Bestandteil dieser Arbeit:
\begin{itemize}
    \item Entwicklung des Fiori-Frontends (wird von anderen Studierenden umgesetzt)
    \item Vollständige Integration mit SAP \gls{calm} (nur konzeptionell)
    \item Benutzer-Authentifizierung (wird durch SAP \gls{btp} bereitgestellt)
\end{itemize}
