\section{Einleitung}

% TODO: Einleitung ausformulieren

\subsection{Motivation und Problemstellung}

Im Rahmen von SAP S/4HANA-Projekten ist das Budget-Controlling ein zentraler Bestandteil der Projektsteuerung. In vielen Unternehmen erfolgt die Budgetplanung und -überwachung noch immer auf Basis von Excel-Dateien, die manuell gepflegt und über SharePoint geteilt werden.

Diese Arbeitsweise bringt mehrere Herausforderungen mit sich:
\begin{itemize}
    \item Keine Echtzeitdaten -- Informationen sind oft veraltet
    \item Manuelle Fehlerquellen bei der Dateneingabe
    \item Fehlende Integration mit SAP-Systemen
    \item Schwierige Nachvollziehbarkeit von Änderungen
    \item Aufwendige manuelle Aggregation von Budget-Metriken
\end{itemize}

Gleichzeitig bietet SAP mit dem \gls{calm} eine moderne Lösung für das Application Lifecycle Management, die perspektivisch auch für das Budget-Controlling genutzt werden könnte. Der Übergang von einer Excel-basierten Lösung zu einer service-orientierten Architektur ist jedoch mit erheblichem Aufwand verbunden.

\subsection{Zielsetzung}

Das Ziel dieser Arbeit ist der Entwurf und die Implementierung einer Middleware, die:
\begin{enumerate}
    \item Eine einheitliche \gls{api} für das Frontend bereitstellt
    \item Aktuell Excel-Daten von SharePoint verarbeiten kann
    \item Zukünftig auf \gls{odata}-Services von SAP \gls{calm} umgestellt werden kann
    \item Den Wechsel zwischen beiden Backends mit minimalem Aufwand ermöglicht
\end{enumerate}

\subsection{Forschungsfrage}

Die zentrale Forschungsfrage dieser Arbeit lautet:

\begin{quote}
\textit{Wie kann eine Middleware-Architektur gestaltet werden, die einen austauschbaren Wechsel zwischen einem Excel-basierten Backend und einer service-orientierten Integration ermöglicht?}
\end{quote}

Daraus ergeben sich folgende Teilfragen:
\begin{itemize}
    \item Welche Architekturmuster eignen sich für austauschbare Datenquellen?
    \item Wie unterscheiden sich die beiden Ansätze hinsichtlich Wartbarkeit und Erweiterbarkeit?
    \item Welcher Implementierungsaufwand ist für den Wechsel zwischen den Backends erforderlich?
\end{itemize}

\subsection{Aufbau der Arbeit}

Die Arbeit gliedert sich wie folgt:

\textbf{Kapitel 2} erläutert die theoretischen Grundlagen zu SAP \gls{cap}, \gls{odata} und relevanten Architekturmustern.

\textbf{Kapitel 3} analysiert die bestehende Situation und definiert die funktionalen sowie nicht-funktionalen Anforderungen an die Middleware.

\textbf{Kapitel 4} beschreibt den Architekturentwurf der Middleware mit Fokus auf die Austauschbarkeit der Backends.

\textbf{Kapitel 5} dokumentiert die Implementierung der Lösung auf der SAP \gls{btp}.

\textbf{Kapitel 6} evaluiert die Ergebnisse und vergleicht beide Backend-Ansätze anhand definierter Kriterien.

\textbf{Kapitel 7} fasst die Ergebnisse zusammen, beantwortet die Forschungsfrage und gibt einen Ausblick auf zukünftige Erweiterungen.
