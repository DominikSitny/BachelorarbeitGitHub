\section{Evaluation}

% TODO: Evaluation ausformulieren mit konkreten Ergebnissen

Dieses Kapitel evaluiert die entwickelte Middleware und vergleicht die beiden Backend-Ansätze (Excel vs. OData-Services) anhand definierter Kriterien.

\subsection{Vergleichskriterien}

Für den Vergleich der Backend-Ansätze werden folgende Kriterien herangezogen:

\begin{enumerate}
    \item \textbf{Implementierungsaufwand}: Aufwand für die initiale Entwicklung
    \item \textbf{Wartbarkeit}: Aufwand für Änderungen und Fehlerbehebung
    \item \textbf{Erweiterbarkeit}: Aufwand für neue Features
    \item \textbf{Performance}: Antwortzeiten und Durchsatz
    \item \textbf{Datenkonsistenz}: Zuverlässigkeit der Daten
    \item \textbf{Austauschbarkeit}: Aufwand für den Backend-Wechsel
\end{enumerate}

\subsection{Vergleich: Excel-Backend vs. OData-Services}

\begin{table}[H]
\centering
\caption{Vergleich der Backend-Ansätze}
\label{tab:vergleich}
\begin{tabularx}{\textwidth}{lXX}
\toprule
\textbf{Kriterium} & \textbf{Excel/SharePoint} & \textbf{OData Services} \\
\midrule
Setup-Aufwand & Gering -- Excel bereits vorhanden & Mittel -- Service-Konfiguration nötig \\
\midrule
Echtzeitdaten & Nein -- Power Query muss manuell aktualisiert werden & Ja -- direkte Abfrage \\
\midrule
Transaktionen & Nein -- keine ACID-Garantien & Ja -- Datenbank-Transaktionen \\
\midrule
Skalierbarkeit & Begrenzt -- Dateigröße limitiert & Hoch -- Datenbank skaliert \\
\midrule
Offline-Fähigkeit & Ja -- Excel lokal verfügbar & Nein -- Service muss erreichbar sein \\
\midrule
Lernkurve Endnutzer & Gering -- Excel bekannt & Höher -- neue Oberfläche \\
\bottomrule
\end{tabularx}
\end{table}

\subsection{Bewertung der Anforderungserfüllung}

\subsubsection{Funktionale Anforderungen}

\begin{table}[H]
\centering
\caption{Erfüllung der funktionalen Anforderungen}
\label{tab:fa-erfuellung}
\begin{tabularx}{\textwidth}{lXc}
\toprule
\textbf{ID} & \textbf{Anforderung} & \textbf{Status} \\
\midrule
FA-01 & Budget-Metriken pro Team berechnen & Erfüllt \\
FA-02 & Budget-Metriken pro \gls{psp} berechnen & Erfüllt \\
FA-03 & \gls{crud} für \gls{psp}-Zuordnungen & Erfüllt \\
FA-04 & Session-basiertes Projekt-Management & Erfüllt \\
FA-05 & Team-Mismatch-Warnungen & Erfüllt \\
FA-06 & Excel-Daten von SharePoint lesen & Teilweise* \\
FA-07 & Änderungen in Excel schreiben & Erfüllt \\
\bottomrule
\end{tabularx}
\end{table}

*FA-06: Lokales Lesen funktioniert, SharePoint-Berechtigungen noch ausstehend.

\subsubsection{Nicht-funktionale Anforderungen}

\begin{table}[H]
\centering
\caption{Erfüllung der nicht-funktionalen Anforderungen}
\label{tab:nfa-erfuellung}
\begin{tabularx}{\textwidth}{lXc}
\toprule
\textbf{ID} & \textbf{Anforderung} & \textbf{Status} \\
\midrule
NFA-01 & Backend austauschbar & Erfüllt \\
NFA-02 & \gls{odata} v4-kompatibel & Erfüllt \\
NFA-03 & Auf SAP \gls{btp} deploybar & Erfüllt \\
NFA-04 & Für Fiori-Frontend geeignet & Erfüllt \\
NFA-05 & Ohne lokale Datenbank & Erfüllt \\
\bottomrule
\end{tabularx}
\end{table}

\subsection{Bewertung der Austauschbarkeit}

Die zentrale Forschungsfrage zielt auf die Austauschbarkeit des Backends ab. Die Evaluation zeigt:

\subsubsection{Durch das Adapter-Pattern geschützte Komponenten}

\begin{itemize}
    \item \gls{cds} Service-Definition -- keine Änderung nötig
    \item Budget-Berechnungslogik -- keine Änderung nötig
    \item Session-Management -- keine Änderung nötig
    \item \gls{api}-Dokumentation -- keine Änderung nötig
\end{itemize}

\subsubsection{Bei Backend-Wechsel zu ändernde Komponenten}

\begin{itemize}
    \item Neuer OData-Adapter implementieren (analog zu Excel-Adapter)
    \item Konfiguration der Service-Endpunkte
    \item Authentifizierung anpassen (OAuth 2.0 für \gls{calm})
\end{itemize}

\subsubsection{Geschätzter Aufwand für Backend-Wechsel}

% TODO: Aufwand schätzen basierend auf Erfahrung

\subsection{Performance-Analyse}

% TODO: Performance-Messungen durchführen und dokumentieren

\begin{table}[H]
\centering
\caption{Antwortzeiten (Beispielmessungen)}
\label{tab:performance}
\begin{tabularx}{\textwidth}{Xcc}
\toprule
\textbf{Endpoint} & \textbf{Datenmenge} & \textbf{Antwortzeit} \\
\midrule
GET /BudgetOverview & 14 Teams & TODO ms \\
GET /UserStories & 251 Stories & TODO ms \\
GET /TimeBookings & 3555 Buchungen & TODO ms \\
\bottomrule
\end{tabularx}
\end{table}

\subsection{Lessons Learned}

% TODO: Erkenntnisse aus der Implementierung

\begin{enumerate}
    \item \textbf{Adapter-Pattern bewährt sich}: Die strikte Trennung zwischen Service- und Adapter-Schicht hat sich als praktikabel erwiesen.

    \item \textbf{Excel-Schreiben komplexer als Lesen}: Das Zurückschreiben in Excel erfordert sorgfältige Handhabung des Formats.

    \item \textbf{\gls{cds} beschleunigt Entwicklung}: Die deklarative Service-Definition reduziert Boilerplate-Code erheblich.

    \item \textbf{Session-Management im RAM riskant}: Bei Serverneustart gehen Sessions verloren -- für Produktion wäre Persistierung nötig.
\end{enumerate}
