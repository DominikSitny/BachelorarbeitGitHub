\section{Evaluation}

In diesem Kapitel wird die entwickelte Middleware evaluiert und die beiden Backend-Ansätze (Excel vs. OData-Services) anhand definierter Kriterien verglichen. Die Evaluation orientiert sich an etablierten Methoden der empirischen Softwaretechnik \citep{wohlin2012} und bewertet sowohl die Anforderungserfüllung als auch qualitative Aspekte der Architektur.

\subsection{Vergleichskriterien}

Für den Vergleich der Backend-Ansätze werden sechs Kriterien herangezogen: der Implementierungsaufwand für die initiale Entwicklung, die Wartbarkeit im Sinne des Aufwands für Änderungen und Fehlerbehebung, die Erweiterbarkeit bezüglich neuer Features, die Performance hinsichtlich Antwortzeiten und Durchsatz, die Datenkonsistenz als Maß für die Zuverlässigkeit der Daten sowie die Austauschbarkeit, also der Aufwand für den Backend-Wechsel.

\subsection{Vergleich: Excel-Backend vs. OData-Services}

\begin{table}[H]
\centering
\caption{Vergleich der Backend-Ansätze}
\label{tab:vergleich}
\begin{tabularx}{\textwidth}{l>{\raggedright\arraybackslash}X>{\raggedright\arraybackslash}X}
\toprule
\textbf{Kriterium} & \textbf{Excel/SharePoint} & \textbf{OData Services} \\
\midrule
Setup-Aufwand & Gering -- Excel bereits vorhanden & Mittel -- Service-Konfiguration nötig \\
\midrule
Echtzeitdaten & Nein -- Power Query muss manuell aktualisiert werden & Ja -- direkte Abfrage \\
\midrule
Transaktionen & Nein -- keine ACID-Garantien & Ja -- Datenbank-Transaktionen \\
\midrule
Skalierbarkeit & Begrenzt -- Dateigröße limitiert & Hoch -- Datenbank skaliert \\
\midrule
Offline-Fähigkeit & Ja -- Excel lokal verfügbar & Nein -- Service muss erreichbar sein \\
\midrule
Lernkurve Endnutzer & Gering -- Excel bekannt & Höher -- neue Oberfläche \\
\bottomrule
\end{tabularx}
\end{table}

\subsection{Bewertung der Anforderungserfüllung}

\subsubsection{Funktionale Anforderungen}

\begin{table}[H]
\centering
\caption{Erfüllung der funktionalen Anforderungen}
\label{tab:fa-erfuellung}
\begin{tabularx}{\textwidth}{l>{\raggedright\arraybackslash}Xc}
\toprule
\textbf{ID} & \textbf{Anforderung} & \textbf{Status} \\
\midrule
FA-01 & Budget-Metriken pro Team berechnen & Erfüllt \\
FA-02 & Budget-Metriken pro \gls{psp} berechnen & Erfüllt \\
FA-03 & \gls{crud} für \gls{psp}-Zuordnungen & Erfüllt \\
FA-04 & Session-basiertes Projekt-Management & Erfüllt \\
FA-05 & Team-Mismatch-Warnungen & Erfüllt \\
FA-06 & Excel-Daten von SharePoint lesen & Teilweise* \\
FA-07 & Änderungen in Excel schreiben & Erfüllt \\
\bottomrule
\end{tabularx}
\end{table}

*FA-06: Lokales Lesen funktioniert, SharePoint-Berechtigungen noch ausstehend.

\subsubsection{Nicht-funktionale Anforderungen}

\begin{table}[H]
\centering
\caption{Erfüllung der nicht-funktionalen Anforderungen}
\label{tab:nfa-erfuellung}
\begin{tabularx}{\textwidth}{l>{\raggedright\arraybackslash}Xc}
\toprule
\textbf{ID} & \textbf{Anforderung} & \textbf{Status} \\
\midrule
NFA-01 & Backend austauschbar & Erfüllt \\
NFA-02 & \gls{odata} v4-kompatibel & Erfüllt \\
NFA-03 & Auf SAP \gls{btp} deploybar & Erfüllt \\
NFA-04 & Für Fiori-Frontend geeignet & Erfüllt \\
NFA-05 & Ohne lokale Datenbank & Erfüllt \\
\bottomrule
\end{tabularx}
\end{table}

\subsection{Bewertung der Austauschbarkeit}

Die zentrale Forschungsfrage zielt auf die Austauschbarkeit des Backends ab. Die Evaluation zeigt:

\subsubsection{Durch das Adapter-Pattern geschützte Komponenten}

Durch das Adapter-Pattern bleiben zentrale Komponenten bei einem Backend-Wechsel unverändert: die \gls{cds} Service-Definition, die Budget-Berechnungslogik, das Session-Management sowie die \gls{api}-Dokumentation erfordern keine Anpassungen.

\subsubsection{Bei Backend-Wechsel zu ändernde Komponenten}

Bei einem Backend-Wechsel müssen hingegen ein neuer OData-Adapter analog zum bestehenden Excel-Adapter implementiert, die Service-Endpunkte konfiguriert und die Authentifizierung auf OAuth 2.0 für \gls{calm} angepasst werden.

\subsubsection{Geschätzter Aufwand für Backend-Wechsel}

Basierend auf der Implementierungserfahrung lässt sich der Aufwand für einen Backend-Wechsel abschätzen:

\begin{table}[H]
\centering
\caption{Geschätzter Aufwand für Backend-Wechsel zu OData}
\label{tab:aufwand}
\begin{tabularx}{\textwidth}{>{\raggedright\arraybackslash}Xlc}
\toprule
\textbf{Komponente} & \textbf{Aufgabe} & \textbf{Aufwand} \\
\midrule
ODataAdapter & Implementierung analog zu ExcelAdapter & 3--5 PT \\
Authentifizierung & OAuth 2.0 für SAP Cloud ALM & 1--2 PT \\
Mapping & Anpassung der Datenstrukturen & 2--3 PT \\
Testing & Integration \& Regressionstests & 2--3 PT \\
\midrule
\textbf{Gesamt} & & \textbf{8--13 PT} \\
\bottomrule
\end{tabularx}
\end{table}

Die Service-Schicht, CDS-Definitionen und das Frontend bleiben dabei unverändert (siehe Tabelle~\ref{tab:aufwand}) -- ein Beleg für die Wirksamkeit des Adapter-Patterns.

\subsection{Performance-Analyse}

Die Performance wurde auf der deployed Instanz (SAP BTP Cloud Foundry, 256 MB RAM) gemessen. Jeder Endpoint wurde 10-mal aufgerufen und der Median der Antwortzeiten ermittelt.

\begin{table}[H]
\centering
\caption{Antwortzeiten der API-Endpunkte}
\label{tab:performance}
\begin{tabularx}{\textwidth}{Xccc}
\toprule
\textbf{Endpoint} & \textbf{Datenmenge} & \textbf{Median} & \textbf{Max} \\
\midrule
GET /BudgetOverview & 14 Teams & 180 ms & 250 ms \\
GET /UserStories & 251 Stories & 320 ms & 480 ms \\
GET /TimeBookings & 3555 Buchungen & 890 ms & 1.200 ms \\
GET /PSPElements & 113 Elemente & 150 ms & 210 ms \\
POST /startSession & -- & 45 ms & 80 ms \\
\bottomrule
\end{tabularx}
\end{table}

\textbf{Bewertung}: Die gemessenen Antwortzeiten liegen größtenteils unter 1 Sekunde, was nach etablierten Usability-Richtlinien als angemessen einzustufen ist -- Antwortzeiten unter 1 Sekunde werden von Nutzern als flüssig wahrgenommen, während Zeiten über 10 Sekunden zu Abbrüchen führen \citep{nielsen1993}. Moderne Web-Performance-Metriken empfehlen Interaktionszeiten unter 200\,ms für eine optimale Benutzererfahrung \citep{google2020}. Der überwiegende Anteil der Latenz ist auf das Parsen der Excel-Datei zurückzuführen. Bei einem Wechsel zu OData-Services ist mit kürzeren Antwortzeiten zu rechnen, da keine Dateitransformation erforderlich ist.

\subsection{Testszenarien}

Die Middleware wurde anhand definierter Testszenarien validiert:

\subsubsection{Funktionale Tests}

\begin{table}[H]
\centering
\caption{Funktionale Testszenarien}
\label{tab:functional-tests}
\begin{tabularx}{\textwidth}{cl>{\raggedright\arraybackslash}Xc}
\toprule
\textbf{ID} & \textbf{Szenario} & \textbf{Erwartetes Ergebnis} & \textbf{Status} \\
\midrule
T-01 & Session starten & Session-ID wird zurückgegeben & \checkmark \\
T-02 & Ungültige Session & 401 Unauthorized & \checkmark \\
T-03 & Budget pro Team abrufen & 14 Teams mit korrekten Summen & \checkmark \\
T-04 & Budget pro PSP abrufen & 113 PSP-Elemente & \checkmark \\
T-05 & OData \$filter anwenden & Gefilterte Ergebnisse & \checkmark \\
T-06 & OData \$top/\$skip & Paginierung funktioniert & \checkmark \\
T-07 & PSP-Zuordnung erstellen & Eintrag in Excel geschrieben & \checkmark \\
T-08 & PSP-Zuordnung löschen & Eintrag aus Excel entfernt & \checkmark \\
T-09 & Team-Mismatch erkennen & Warnung wird generiert & \checkmark \\
T-10 & Projekt wechseln & Neue Session für anderes Projekt & \checkmark \\
\bottomrule
\end{tabularx}
\end{table}

\subsubsection{Grenzwert-Tests}

\begin{table}[H]
\centering
\caption{Grenzwert-Testszenarien}
\label{tab:boundary-tests}
\begin{tabularx}{\textwidth}{l>{\raggedright\arraybackslash}Xc}
\toprule
\textbf{Szenario} & \textbf{Erwartetes Verhalten} & \textbf{Status} \\
\midrule
Leere Excel-Datei & Leere Arrays werden zurückgegeben & \checkmark \\
User Story ohne PSP & pspElement = \enquote{unassigned}, nicht in ETC & \checkmark \\
PSP ohne Budget & budgetPT = 0 & \checkmark \\
Doppelte PSP-Zuordnung & 409 Conflict & \checkmark \\
Session nach 8h & 401 Session abgelaufen & \checkmark \\
Ungültiges Projekt & 404 Not Found & \checkmark \\
\bottomrule
\end{tabularx}
\end{table}

\subsection{SWOT-Analyse der Architektur}

Eine SWOT-Analyse (Strengths, Weaknesses, Opportunities, Threats) ist ein etabliertes Instrument der strategischen Planung, das sowohl interne Faktoren (Stärken, Schwächen) als auch externe Faktoren (Chancen, Risiken) berücksichtigt \citep{helms2010}. Die folgende Analyse fasst die Bewertung der gewählten Architektur zusammen:

\begin{table}[H]
\centering
\caption{SWOT-Analyse der BC Middleware Architektur}
\label{tab:swot}
\begin{tabularx}{\textwidth}{|X|X|}
\hline
\textbf{Stärken (Strengths)} & \textbf{Schwächen (Weaknesses)} \\
\hline
\begin{itemize}[leftmargin=*, nosep]
    \item Austauschbares Backend durch Adapter-Pattern
    \item OData-Konformität für Fiori-Integration
    \item Keine zusätzliche Datenbank erforderlich
    \item Einfaches Deployment auf SAP BTP
    \item Excel als vertraute Datenquelle für Anwender
\end{itemize}
&
\begin{itemize}[leftmargin=*, nosep]
    \item Keine Echtzeitdaten (Excel-Refresh nötig)
    \item Sessions nicht persistent
    \item Keine ACID-Transaktionen
    \item Begrenzte Skalierbarkeit bei großen Excel-Dateien
    \item SharePoint-Integration noch nicht produktiv
\end{itemize}
\\
\hline
\textbf{Chancen (Opportunities)} & \textbf{Risiken (Threats)} \\
\hline
\begin{itemize}[leftmargin=*, nosep]
    \item Migration zu SAP Cloud ALM möglich
    \item Erweiterung um weitere Projektcontrolling-Funktionen
    \item Integration mit SAP Analytics Cloud
    \item Wiederverwendung des Patterns für andere Projekte
\end{itemize}
&
\begin{itemize}[leftmargin=*, nosep]
    \item Excel-Datei wird versehentlich beschädigt
    \item Konkurrierende Zugriffe auf Excel
    \item API-Änderungen bei SAP Cloud ALM
    \item Performance bei sehr großen Projekten
\end{itemize}
\\
\hline
\end{tabularx}
\end{table}

\subsection{Lessons Learned}

Aus der Implementierung und dem Betrieb der Middleware ergeben sich mehrere Erkenntnisse.

Das Adapter-Pattern hat sich bewährt: Die strikte Trennung zwischen Service- und Adapter-Schicht erwies sich als praktikabel. Änderungen am Excel-Format erforderten nur Anpassungen im Adapter, nicht in der Geschäftslogik, was die Empfehlungen von \cite{gamma1994} bestätigt.

Das Schreiben in Excel erwies sich als deutlich komplexer als das Lesen. Insbesondere Formeln, Formatierungen und Zellreferenzen müssen erhalten bleiben. Die xlsx-Bibliothek unterstützt dies, erfordert aber detailliertes Verständnis der Excel-Interna.

Die deklarative Service-Definition mit \gls{cds} beschleunigt die Entwicklung erheblich, da Boilerplate-Code reduziert wird und OData-Konformität, Filterung sowie Paginierung automatisch bereitgestellt werden \citep{sap2024cap}.

Das Session-Management im RAM birgt Risiken: Bei Serverneustart gehen Sessions verloren, weshalb für den produktiven Einsatz eine Persistierung etwa über Redis oder eine Datenbank erforderlich wäre \citep{richardson2018}.

Schließlich zeigt sich, dass die Middleware als Integrationsmuster fungiert -- konkret als \enquote{Anti-Corruption Layer} \citep{vernon2013}, der das Frontend vor den Eigenheiten der Excel-Struktur schützt. Dieses Muster ist auch aus den Enterprise Integration Patterns bekannt, wo es als \enquote{Message Translator} beschrieben wird \citep{hohpe2003}.
