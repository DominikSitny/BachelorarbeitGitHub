\section{Fazit und Ausblick}

Dieses abschließende Kapitel fasst die Ergebnisse der Arbeit zusammen, beantwortet die Forschungsfrage und gibt einen Ausblick auf zukünftige Entwicklungsmöglichkeiten.

\subsection{Zusammenfassung}

Diese Bachelorarbeit hat den Entwurf und die Implementierung einer Middleware für Budget-Controlling im SAP-Umfeld dokumentiert. Die entwickelte BC Middleware erfüllt die gestellten Anforderungen:

\begin{itemize}
    \item Eine einheitliche \gls{odata} v4-konforme \gls{api} wurde bereitgestellt
    \item Excel-Daten werden erfolgreich geparst und aggregiert
    \item Budget-Metriken (Budget, Actual, \gls{etc}, \gls{eac}) werden korrekt berechnet
    \item \gls{crud}-Operationen für \gls{psp}-Zuordnungen wurden implementiert
    \item Die Architektur ermöglicht den Austausch des Backends
\end{itemize}

Die Middleware wurde erfolgreich auf der SAP \gls{btp} deployed und steht für die Integration mit dem Fiori-Frontend bereit.

\subsection{Beantwortung der Forschungsfrage}

Die zentrale Forschungsfrage lautete:

\begin{quote}
\textit{Wie kann eine Middleware-Architektur gestaltet werden, die einen austauschbaren Wechsel zwischen einem Excel-basierten Backend und einer service-orientierten Integration ermöglicht?}
\end{quote}

Es wird gezeigt, dass durch konsequente Anwendung des \textbf{Adapter-Patterns} eine solche Architektur realisierbar ist. Die Schlüsselelemente umfassen:

\begin{enumerate}
    \item \textbf{Einheitliches Datenmodell}: Unabhängig von der Datenquelle wird intern das gleiche Modell verwendet.

    \item \textbf{Adapter-Schicht}: Die Adapter-Schicht kapselt alle datenquellenspezifischen Details. Die Service-Schicht arbeitet nur mit dem abstrakten Interface.

    \item \textbf{Deklarative Service-Definition}: SAP \gls{cap} mit \gls{cds} ermöglicht eine saubere Trennung zwischen \gls{api}-Definition und Implementierung.

    \item \textbf{Berechnung statt Persistierung}: Die Budget-Metriken werden zur Laufzeit berechnet, nicht gespeichert. Dies vereinfacht den Backend-Wechsel.
\end{enumerate}

Der geschätzte Aufwand für einen Wechsel von Excel zu \gls{odata}-Services besteht primär in der Implementierung eines neuen Adapters. Die Service-Schicht, \gls{api}-Definition und das Frontend bleiben unverändert.

\subsection{Kritische Reflexion}

Eine wissenschaftliche Arbeit erfordert auch die kritische Betrachtung der eigenen Ergebnisse \citep{wohlin2012}. Folgende Einschränkungen sind zu berücksichtigen:

\begin{itemize}
    \item \textbf{SharePoint-Integration}: Die direkte SharePoint-Anbindung konnte aufgrund fehlender Berechtigungen nicht vollständig getestet werden. Die lokale Simulation mit Dateien im \texttt{excel\_templates/}-Verzeichnis ist funktional äquivalent, aber die Latenz und Fehlerbehandlung bei Netzwerkzugriffen wurden nicht evaluiert.

    \item \textbf{Session-Persistierung}: Sessions werden nur im RAM gehalten und gehen bei Serverneustart verloren. Für den produktiven Einsatz ist dies eine relevante Einschränkung, die durch eine persistente Session-Speicherung adressiert werden müsste.

    \item \textbf{\gls{crud}-Tests}: Die Write-Back-Funktionalität für \gls{psp}-Zuordnungen wurde mit begrenzten Testdaten validiert. Ausführliche Tests mit größeren Datenmengen und konkurrierenden Zugriffen stehen noch aus.

    \item \textbf{Konzeptioneller OData-Adapter}: Der OData-Adapter für SAP \gls{calm} wurde nur konzeptionell beschrieben, nicht implementiert. Der tatsächliche Aufwand für den Backend-Wechsel kann daher von der Schätzung abweichen.

    \item \textbf{Performance unter Last}: Die Performance-Messungen erfolgten mit einzelnen sequentiellen Anfragen. Das Verhalten unter Last (mehrere gleichzeitige Benutzer) wurde nicht systematisch untersucht.
\end{itemize}

Diese Einschränkungen beeinträchtigen nicht die Aussagekraft der Architekturentscheidungen, sind jedoch bei der Weiterentwicklung zu berücksichtigen.

\subsection{Ausblick}

Für die Weiterentwicklung der BC Middleware ergeben sich folgende Möglichkeiten:

\subsubsection{Kurzfristig}

\begin{itemize}
    \item Umfangreiche Tests der \gls{crud}-Funktionalität
    \item Aktivierung der SharePoint-Integration nach Erhalt der Berechtigungen
    \item Performance-Optimierung durch Caching
\end{itemize}

\subsubsection{Mittelfristig}

\begin{itemize}
    \item Implementierung des OData-Adapters für SAP \gls{calm}
    \item Persistierung der Sessions (z.B. in Redis)
    \item Erweiterung um Historisierung von Änderungen
\end{itemize}

\subsubsection{Langfristig}

\begin{itemize}
    \item Vollständige Migration auf SAP \gls{calm} als Datenquelle
    \item Integration mit SAP Analytics Cloud für Reporting
    \item Erweiterung auf weitere Projektcontrolling-Funktionen
\end{itemize}

\subsection{Fazit}

Die BC Middleware demonstriert, dass eine austauschbare Backend-Architektur im SAP-Umfeld praktikabel umsetzbar ist. Das Adapter-Pattern in Kombination mit SAP \gls{cap} bildet eine tragfähige Grundlage für die schrittweise Migration von Excel-basierten Lösungen zu service-orientierten Architekturen.

Die Arbeit liefert damit einen Beitrag zur Frage, wie Unternehmen ihre bestehenden Excel-Workflows modernisieren können, ohne dabei einen abrupten Systemwechsel vollziehen zu müssen. Angesichts der in der Literatur dokumentierten Fehleranfälligkeit von Spreadsheet-Anwendungen \citep{powell2009, panko2016} und der wachsenden Bedeutung der digitalen Transformation \citep{westerman2014} ist dieser schrittweise Migrationsansatz von praktischer Relevanz für viele Unternehmen.
