\section{Fazit und Ausblick}

Dieses abschließende Kapitel fasst die Ergebnisse der Arbeit zusammen, beantwortet die Forschungsfrage und gibt einen Ausblick auf zukünftige Entwicklungsmöglichkeiten.

\subsection{Zusammenfassung}

Ausgangspunkt dieser Arbeit war ein Excel-basierter Workflow im Budget-Controlling, der unter fehlender Echtzeitfähigkeit, fehlender Programmierschnittstelle und mangelnder Konsistenzprüfung litt (vgl. Abschnitt~3.1.3). Ziel war der Entwurf und die Implementierung einer Middleware, die diese Probleme adressiert und gleichzeitig einen späteren Wechsel zu service-orientierten Backends ermöglicht.

Die entwickelte BC Middleware löst die identifizierten Probleme durch eine OData v4-konforme \gls{api}, die auf SAP \gls{cap} basiert und sämtliche Budget-Daten über standardisierte Endpunkte bereitstellt. Die vier Metriken Budget, Actual, \gls{etc} und \gls{eac} werden zur Laufzeit aus den Excel-Daten berechnet und in zwei Aggregationsebenen (pro Team und pro \gls{psp}) angeboten. Der PSP-Override-Mechanismus stellt sicher, dass manuelle Zuordnungen über die \gls{api} sofort wirksam werden. Insgesamt wurden sechs von sieben funktionalen und alle fünf nicht-funktionalen Anforderungen vollständig erfüllt (vgl. Abschnitt~6.3). Die Middleware ist auf der SAP \gls{btp} deployed, durch 113 automatisierte Tests abgesichert und steht für die Integration mit dem Fiori-Frontend bereit.

\subsection{Beantwortung der Forschungsfrage}

Die zentrale Forschungsfrage lautete:

\begin{quote}
\textit{Wie kann eine Cloud-Middleware zur Abstraktion heterogener Datenquellen auf der SAP BTP konzipiert werden, sodass ein Wechsel zwischen einem Excel-basierten Backend und einer service-orientierten Integration mit minimalem Aufwand möglich ist?}
\end{quote}

Die Arbeit zeigt, dass durch konsequente Anwendung des \textbf{Adapter-Patterns} \citep{gamma1994} eine solche Architektur realisierbar ist. Die Antwort gliedert sich in drei Aspekte.

Erstens ermöglicht ein \textbf{einheitliches Datenmodell}, das in \gls{cds} deklarativ definiert wird, die Entkopplung der \gls{api}-Schnittstelle von der konkreten Datenquelle. Die Entitäten (\texttt{Teams}, \texttt{PSPElements}, \texttt{BudgetOverview} etc.) werden mit \texttt{@cds.persistence.skip} annotiert, sodass SAP \gls{cap} einen vollständigen OData-Service generiert, ohne eine Datenbank vorauszusetzen. Dieses Modell bleibt bei einem Backend-Wechsel unverändert.

Zweitens kapselt die \textbf{Adapter-Schicht} alle datenquellenspezifischen Details hinter einer einheitlichen Schnittstelle. Der Excel-Adapter übersetzt die heterogenen Sheet-Formate der \texttt{Budgetcontrolling.xlsx} in das interne Datenmodell; ein zukünftiger OData-Adapter würde die gleiche Schnittstelle mit Daten aus SAP \gls{calm} bedienen. Die Service-Schicht, die Budget-Berechnungslogik und das Session-Management arbeiten ausschließlich mit dem abstrakten Interface und haben keine Kenntnis über die konkrete Datenquelle.

Drittens vereinfacht die \textbf{Laufzeitberechnung} der Budget-Metriken den Backend-Wechsel zusätzlich: Da keine Daten persistiert werden, entfallen Migrationsprobleme. Die Metriken werden bei jedem Request aus den aktuell geladenen Daten berechnet, unabhängig von deren Herkunft.

Der geschätzte Aufwand für einen vollständigen Wechsel von Excel zu OData-Services liegt bei 8--13 Personentagen (vgl. Tabelle~\ref{tab:aufwand}), wobei über 80\,\% der bestehenden Codebasis unverändert bleiben. Dies bestätigt die Wirksamkeit des gewählten Architekturansatzes.

\subsection{Kritische Reflexion}

Eine wissenschaftliche Arbeit erfordert auch die kritische Betrachtung der eigenen Ergebnisse \citep{wohlin2012}. Folgende Einschränkungen sind zu berücksichtigen:

Die direkte SharePoint-Anbindung konnte aufgrund fehlender Berechtigungen nicht vollständig getestet werden. Die lokale Simulation mit Dateien im \texttt{excel\_templates/}-Verzeichnis ist funktional äquivalent, jedoch wurden die Latenz und Fehlerbehandlung bei Netzwerkzugriffen nicht evaluiert.

Sessions werden ausschließlich im RAM gehalten und gehen bei Serverneustart verloren. Für den produktiven Einsatz ist dies eine relevante Einschränkung, die durch eine persistente Session-Speicherung adressiert werden müsste.

Die Write-Back-Funktionalität für \gls{psp}-Zuordnungen wurde mit begrenzten Testdaten validiert. Ausführliche Tests mit größeren Datenmengen und konkurrierenden Zugriffen stehen noch aus. Ebenso wurde der OData-Adapter für SAP \gls{calm} nur konzeptionell beschrieben, nicht implementiert, weshalb der tatsächliche Aufwand für den Backend-Wechsel von der Schätzung abweichen kann.

Schließlich erfolgten die Performance-Messungen mit einzelnen sequentiellen Anfragen. Das Verhalten unter Last mit mehreren gleichzeitigen Benutzern wurde nicht systematisch untersucht.

Diese Einschränkungen beeinträchtigen nicht die Aussagekraft der Architekturentscheidungen, sind jedoch bei der Weiterentwicklung zu berücksichtigen.

\subsection{Ausblick}

Für die Weiterentwicklung der BC Middleware ergeben sich folgende Möglichkeiten:

\subsubsection{Kurzfristig}

Kurzfristig sollten die \gls{crud}-Operationen für \gls{psp}-Zuordnungen umfangreich mit größeren Datenmengen und konkurrierenden Zugriffen getestet werden, da die bisherige Validierung auf begrenzten Testdaten basiert. Nach Erhalt der SharePoint-Berechtigungen kann die bestehende Microsoft-Graph-Integration aktiviert werden, sodass die Middleware die \texttt{Budgetcontrolling.xlsx} direkt von SharePoint lädt, anstatt auf lokale Kopien zurückzugreifen. Zusätzlich könnte ein zeitbasierter Cache die Performance verbessern, indem die Excel-Datei nicht bei jedem Request neu geparst wird, sondern erst nach einer konfigurierbaren Zeitspanne.

\subsubsection{Mittelfristig}

Mittelfristig steht die Implementierung des OData-Adapters für SAP \gls{calm} an, der die zentrale Architekturentscheidung dieser Arbeit validieren würde. Die geschätzten 8--13 Personentage (vgl. Tabelle~\ref{tab:aufwand}) könnten durch die bereits vorhandene Adapter-Schnittstelle und die unveränderte Service-Schicht realisiert werden. Ergänzend sollten die Sessions über einen externen Speicher wie Redis persistiert werden, um den Verlust bei Serverneustart zu vermeiden \citep{richardson2018}. Eine Historisierung von Änderungen an \gls{psp}-Zuordnungen würde die Nachvollziehbarkeit verbessern und das in der Problemanalyse (Abschnitt~3.1.3) identifizierte vierte Problem vollständig adressieren.

\subsubsection{Langfristig}

Langfristig ist eine vollständige Migration auf SAP \gls{calm} als alleinige Datenquelle denkbar, bei der der Excel-Adapter zugunsten des OData-Adapters entfernt wird. In diesem Szenario entfallen der PSP-Override-Mechanismus und die Write-Back-Funktionalität, da Änderungen direkt im Backend-System persistiert würden. Eine Integration mit SAP Analytics Cloud könnte das Reporting erweitern, und das Architekturmuster der Middleware ließe sich auf weitere Projektcontrolling-Funktionen oder andere Domänen übertragen.

\subsection{Fazit}

Die BC Middleware demonstriert, dass eine austauschbare Backend-Architektur im SAP-Umfeld praktikabel umsetzbar ist. Das Adapter-Pattern in Kombination mit SAP \gls{cap} bildet eine tragfähige Grundlage für die schrittweise Migration von Excel-basierten Lösungen zu service-orientierten Architekturen. Die Tatsache, dass über 80\,\% der Codebasis bei einem Backend-Wechsel unverändert bleiben, belegt die Wirksamkeit dieses Ansatzes.

Die Arbeit liefert damit einen Beitrag zur Frage, wie Unternehmen ihre bestehenden Excel-Workflows modernisieren können, ohne dabei einen abrupten Systemwechsel vollziehen zu müssen. Angesichts der in der Literatur dokumentierten Fehleranfälligkeit von Spreadsheet-Anwendungen \citep{powell2009, panko2016} und der wachsenden Bedeutung der digitalen Transformation \citep{westerman2014} ist dieser schrittweise Migrationsansatz -- im Sinne des Strangler-Fig-Prinzips \citep{newman2021} -- von praktischer Relevanz für viele Unternehmen, die sich in vergleichbaren Migrationssituationen befinden.
