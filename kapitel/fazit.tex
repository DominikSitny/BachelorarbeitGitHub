\section{Fazit und Ausblick}

Dieses abschließende Kapitel fasst die Ergebnisse der Arbeit zusammen, beantwortet die Forschungsfrage und gibt einen Ausblick auf zukünftige Entwicklungsmöglichkeiten.

\subsection{Zusammenfassung}

Diese Bachelorarbeit hat den Entwurf und die Implementierung einer Middleware für Budget-Controlling im SAP-Umfeld dokumentiert. Die entwickelte BC Middleware erfüllt die gestellten Anforderungen: Es wurde eine einheitliche \gls{odata} v4-konforme \gls{api} bereitgestellt, über die Excel-Daten erfolgreich geparst und aggregiert werden. Die Budget-Metriken (Budget, Actual, \gls{etc}, \gls{eac}) werden korrekt berechnet und \gls{crud}-Operationen für \gls{psp}-Zuordnungen wurden implementiert. Die gewählte Architektur ermöglicht darüber hinaus den Austausch des Backends.

Die Middleware wurde erfolgreich auf der SAP \gls{btp} deployed und steht für die Integration mit dem Fiori-Frontend bereit.

\subsection{Beantwortung der Forschungsfrage}

Die zentrale Forschungsfrage lautete:

\begin{quote}
\textit{Wie kann eine Cloud-Middleware zur Abstraktion heterogener Datenquellen auf der SAP BTP konzipiert werden, sodass ein Wechsel zwischen einem Excel-basierten Backend und einer service-orientierten Integration mit minimalem Aufwand möglich ist?}
\end{quote}

Es wird gezeigt, dass durch konsequente Anwendung des \textbf{Adapter-Patterns} eine solche Architektur realisierbar ist. Die Schlüsselelemente dieser Architektur sind ein einheitliches Datenmodell, das unabhängig von der Datenquelle intern verwendet wird, sowie eine Adapter-Schicht, die alle datenquellenspezifischen Details kapselt, sodass die Service-Schicht ausschließlich mit dem abstrakten Interface arbeitet. Die deklarative Service-Definition mit SAP \gls{cap} und \gls{cds} ermöglicht eine saubere Trennung zwischen \gls{api}-Definition und Implementierung. Schließlich werden die Budget-Metriken zur Laufzeit berechnet statt persistiert, was den Backend-Wechsel zusätzlich vereinfacht.

Der geschätzte Aufwand für einen Wechsel von Excel zu \gls{odata}-Services besteht primär in der Implementierung eines neuen Adapters. Die Service-Schicht, \gls{api}-Definition und das Frontend bleiben unverändert.

\subsection{Kritische Reflexion}

Eine wissenschaftliche Arbeit erfordert auch die kritische Betrachtung der eigenen Ergebnisse \citep{wohlin2012}. Folgende Einschränkungen sind zu berücksichtigen:

Die direkte SharePoint-Anbindung konnte aufgrund fehlender Berechtigungen nicht vollständig getestet werden. Die lokale Simulation mit Dateien im \texttt{excel\_templates/}-Verzeichnis ist funktional äquivalent, jedoch wurden die Latenz und Fehlerbehandlung bei Netzwerkzugriffen nicht evaluiert.

Sessions werden ausschließlich im RAM gehalten und gehen bei Serverneustart verloren. Für den produktiven Einsatz ist dies eine relevante Einschränkung, die durch eine persistente Session-Speicherung adressiert werden müsste.

Die Write-Back-Funktionalität für \gls{psp}-Zuordnungen wurde mit begrenzten Testdaten validiert. Ausführliche Tests mit größeren Datenmengen und konkurrierenden Zugriffen stehen noch aus. Ebenso wurde der OData-Adapter für SAP \gls{calm} nur konzeptionell beschrieben, nicht implementiert, weshalb der tatsächliche Aufwand für den Backend-Wechsel von der Schätzung abweichen kann.

Schließlich erfolgten die Performance-Messungen mit einzelnen sequentiellen Anfragen. Das Verhalten unter Last mit mehreren gleichzeitigen Benutzern wurde nicht systematisch untersucht.

Diese Einschränkungen beeinträchtigen nicht die Aussagekraft der Architekturentscheidungen, sind jedoch bei der Weiterentwicklung zu berücksichtigen.

\subsection{Ausblick}

Für die Weiterentwicklung der BC Middleware ergeben sich folgende Möglichkeiten:

\subsubsection{Kurzfristig}

Kurzfristig sollten umfangreiche Tests der \gls{crud}-Funktionalität durchgeführt, die SharePoint-Integration nach Erhalt der Berechtigungen aktiviert und die Performance durch Caching optimiert werden.

\subsubsection{Mittelfristig}

Mittelfristig steht die Implementierung des OData-Adapters für SAP \gls{calm} an, ergänzt durch die Persistierung der Sessions etwa über Redis sowie die Erweiterung um eine Historisierung von Änderungen.

\subsubsection{Langfristig}

Langfristig ist eine vollständige Migration auf SAP \gls{calm} als Datenquelle denkbar, verbunden mit einer Integration in SAP Analytics Cloud für das Reporting und einer Erweiterung auf weitere Projektcontrolling-Funktionen.

\subsection{Fazit}

Die BC Middleware demonstriert, dass eine austauschbare Backend-Architektur im SAP-Umfeld praktikabel umsetzbar ist. Das Adapter-Pattern in Kombination mit SAP \gls{cap} bildet eine tragfähige Grundlage für die schrittweise Migration von Excel-basierten Lösungen zu service-orientierten Architekturen.

Die Arbeit liefert damit einen Beitrag zur Frage, wie Unternehmen ihre bestehenden Excel-Workflows modernisieren können, ohne dabei einen abrupten Systemwechsel vollziehen zu müssen. Angesichts der in der Literatur dokumentierten Fehleranfälligkeit von Spreadsheet-Anwendungen \citep{powell2009, panko2016} und der wachsenden Bedeutung der digitalen Transformation \citep{westerman2014} ist dieser schrittweise Migrationsansatz von praktischer Relevanz für viele Unternehmen.
