\documentclass[12pt, captions=tableabove]{scrartcl}

\usepackage{fontspec} % Native Schriftwahl & UTF-8
\setmainfont{Arial}[SmallCapsFont={Latin Modern Roman}]
\setkomafont{disposition}{\normalfont\bfseries}

\usepackage{geometry}
\geometry{left=3cm,right=2cm,top=2cm,bottom=2cm, footskip=0.8cm, head=22pt}
\setlength{\footheight}{22pt}

\usepackage[none]{hyphenat} % Silbentrennung deaktivieren
\sloppy % Erlaubt LaTeX, beim Blocksatz etwas großzügiger zu arbeiten

\usepackage{microtype} % Aktiviert das Micro-Typography-Paket

\renewcommand{\baselinestretch}{1.25} % Zeilenabstand einstellen %
\usepackage[dvipsnames]{xcolor} % z. B. Schriftfarbe ändern

\usepackage{tabularx} % Im Vergleich zur Standardtabellen gibt es mit tabularx die Möglichkeit der Tabelle eine bestimmte Breite vorzugeben.

\usepackage[nopostdot]{glossaries} % Package für das Abkürzungsverzeichnis

\usepackage[utf8]{inputenc} % Unterstützung von UTF-8 für ältere LaTeX Versionen und Overleaf

\usepackage[english,main=ngerman]{babel} % Die zuletzte angeben Sprache ist die zu Beginn bereitgestellte Sprache. Sprache wechseln mit z. B. \selectlanguage{german}

%------- Einstellung der Kopf- und Fußzeilen. Kompletten Block löschen, wenn nicht benötigt.
\usepackage[automark,headsepline,markcase=used]{scrlayer-scrpage} % Einfügen der "headsepline" mit der Kapitel Überschrift

\clearpairofpagestyles

% Kopfzeile
\ihead{\leftmark}        % Innen (bei zweiseitigem Layout: links/rechts wechselnd): Kapitel oder Abschnitt
\ohead{}                 % Außen: leer

% Fußzeile
\ifoot{} % Innen: leer
\cfoot{}                 % Mitte: leer
\ofoot{\pagemark}        % Außen: Seitenzahl

\usepackage[hang, flushmargin]{footmisc}  % Fußnoten-Layout anpassen
\setlength{\footnotemargin}{1em}          % Abstand zwischen Nummer und Text

% Seitenstil aktivieren
\pagestyle{scrheadings}
%------- Ende der Einstellung der Kopf- und Fußzeilen.

\usepackage{graphicx} % Das "graphicsx" Package ist z. B. für \includegraphics, also um Bilder einzufügen

\usepackage{float} % Das "float" Package ist z. B. für den "[H]" Befehl bei "/begin{figure}", dieser bewirkt, dass die Bilder an der gewünschten Stelle angezeigt werden.

\usepackage[margin=0pt,font=footnotesize,labelfont=bf]{caption}
\captionsetup[figure]{singlelinecheck=false}
\setlength\parindent{0pt} % Keine Whitespaces im Text nach einer Figure

\usepackage{longtable} % Tabellen darstellen, die über mehrere Seiten gehen

\usepackage{amsmath} % Mathematischen Befehle darstellen

\usepackage{tikz} % Paket für Diagramme und Grafiken
\usetikzlibrary{shapes.geometric, arrows, positioning}

\usepackage{booktabs} % Professionelle Tabellenlinien (toprule, midrule, bottomrule)
\usepackage{enumitem} % Anpassung von Listen (nosep, leftmargin)
\usepackage{multirow} % Mehrzeilige Tabellenzellen

\usepackage[autostyle=true,german=quotes]{csquotes} % Deutsche Anführungszeichen mit \enquote{}

\usepackage{listings} % Paket für Code Auflistungen

\usepackage{pythonhighlight} % Python Code

\renewcommand{\lstlistingname}{Auflistung} % Code Listing -> Auflistung

\usepackage{natbib} % Paket, um das Zitieren zu ermöglichen

\usepackage{url} % Paket, um URLs einzubinden
\usepackage{hyperref} % Verlinkungen 
\hypersetup{
  colorlinks=true,
  linkcolor=black,
  citecolor=black,
  urlcolor = blue,
} % URLs Blau, Rest Schwarz

\lstdefinestyle{mystyle}{
basicstyle=\small\sffamily,
numbers=left,
numberstyle=\tiny,
frame=tb,
columns=fullflexible,
showstringspaces=false
}
\lstset{style=mystyle}

% Führt Füllpunkte auch für manuelle \addcontentsline-Einträge im Inhaltsverzeichnis ein
\DeclareTOCStyleEntry[
  indent=0pt,
  linefill=\TOCLineLeaderFill
]{tocline}{section}

\RedeclareSectionCommand[
  beforeskip=-.5\baselineskip,
  afterskip=.25\baselineskip]{paragraph} % Newline nach "paragraph" Überschrift